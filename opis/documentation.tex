\documentclass[a4paper]{article}
\usepackage[utf8]{inputenc}
\usepackage[polish]{babel}
\usepackage[T1]{fontenc}
\usepackage{indentfirst}
\usepackage{amsmath, amsfonts}
\usepackage[nofoot,hdivide={2cm,*,2cm},vdivide={2cm,*,2cm}]{geometry}
\frenchspacing
\pagestyle{empty}

\author{Franciszek Zdobylak}
\title{\Huge{\bf{Gra w Statki}}\\
				\small Dokumentacja}

\begin{document}
\maketitle

\section{Komunikacja między procesami}
    Programy komunikują się przy użyciu plików kolejkowych. Do ich implementacji użyłem pliku pokazywanego podczas wykładu 
    Wstępu do C (\texttt{lin-fifo.c}). 

    Do obsługi użyłem swojego modułu, który wysyła specjalnie przygotowane wiadomości do bliźniaczego procesu używając
interfejsu wytworzonego w poprzednio wspomnianym pliku. 

Wszystkie wiadomości są postaci $mx_{1}x_{2}s$, gdzie \\
$m$ -- tryb (s - zapytanie, f - odpowiedź, n - nowa gra, g - poddanie się), \\
$x_{1},x_{2}$ -- współrzędnie strzału (jeśli wiadomość nie wymaga tego to dowolne znaki), \\
$s$ -- status pola oznaczonego przez współrzędne $x_{1}, x_{2}$. 
Dostępne funkcje:
\begin{itemize}
    \item \texttt{sendMove(PipesPtr, Shoot)} - wysyła zapytanie o dane pole na planszy
    \item \texttt{sendFeedback(PipesPtr, Shoot, Status)} - wysyła odpowiedź na zapytanie poprzedniej funkcji
    \item \texttt{sendSignal(PipesPtr, int)} - wysyłanie sygnału nowej gry(1), poddania się (0)
    \item \texttt{sendReveal(PipesPtr, Shoot, Status)} - wysyłanie inforamcji o polu na planszy po zakończeniu gry (odkrywanie planszy)
    \item \texttt{getMessage(PipesPtr, char*)} - odbieranie wiadomości i zapisywanie w stringu podanym jako parametr
    \end{itemize}

\section{Logika gry}
    Do obsługi gry służy moduł znajdujący się w plikach \texttt{game.c i game.h}. 
Są w nim opisane wszystkie funkcje oraz typy danych potrzebne do toczenia rozgrywki.

\subsection{Typy danych}
\begin{itemize}
    \item Typ wyliczeniowy \texttt{Status} - służy do zaznaczania statnów na planszy. Składa się z elementów:
    NOT\_SHOOT, SHIP, MISSED, HIT, SUNK, MY\_HIT, UNKNOWN.
    \item Struktura \texttt{Shoot} - para liczb, służy do określenie pozycji na planszy
    \item Struktura \texttt{Ships} - przechowuje ilość statków poszczególnej długości, 
        długość najdłuższego oraz liczbę wszystkich statków i pozostałych (niezestrzelonych)
    \item Typ \texttt{Board} - tablica 2-wymiarowa 10x10, przechowuje opis planszy (zgodnie z wartościami typu \texttt{Status})
\end{itemize}

\subsection{Ważniejsze funkcje}
\begin{itemize}
    \item \texttt{bool isSunk(Shoot s, Board b)} - zwraca prawdę jeśli statek, którego część znajduje się
na pozycji podanej jako pierwszy argument, jest zatopiony (tzn. każde pole na którym się znajduje oznaczone jest
jako trafione
    \item \texttt{int markSunk(Shoot s, Board b)} - zaznacza cały statek jako zatopiony (zmienia status z HIT/MY\_HIT na SUNK),
zwraca długość zaznaczonego statku
    \item \texttt{bool placeShip(int length, int orientation, Shoot, Board)} - próbuje ustawić statek
o długości \texttt{length} i orientacji (1 - poziomej, 0 - pionowej) zaczynając na polu s na planszy b.
Jeśli się da to zwraca prawdę i zaznacza statek na planszy. W przeciwnym wypadku zwraca fałsz.
    \item \texttt{void removeShip(Shoot pos, Board b)} - usuwa statek którego część znajduje się na pozycji \texttt{pos} z planszy b.
\end{itemize}

\section{Tworzenie planszy}
Tworzenie plansz oraz ładowanie ilości statków z pliku umieszczone jest w module w plikach \texttt{boards.c i boards.h}. 
Dostępne funkcje:
\begin{itemize}
    \item \texttt{getShips(Ships *s)} - pobiera ilość statków z pliku \texttt{ships.len}. Przy wczytywaniu
pomija linie oznaczone znakiem \texttt{\#} (są to komentarze).
    \item \texttt{randBoard(Board b, Ships *s, char name)} - funkcja wywoływana przez resztę modułów.
Generuje planszę oraz wczytuje długości statków z pliku.
    \item \texttt{genBoard(Board b, Ships s, char name)} - funkcja generująca plnnaszę z ilością statków,
które zostały podane jako jeden z parametrów. Znak \texttt{name} służy do lepszego generowania losowych liczb. 
Algorytm wstawiania statków jest prosty. Losujemy miejsce w którym statek ma się zaczynać oraz jego orientację. 
Jeśli statek się w takim miejscu nie zmieści to sprawdzamy czy zmieści się w innej orientacji w tym miejscu. 
Jeśli znów nie, to przesuwamy się o jedno miejsce dalej. Jeśli statek nie może zostać położony w żadny miejscu,
to generujemy planszę od nowa. Statków nie jest za dużo, więc praktycznie jest to niemożliwe aby algorytm się zapętlił w nieskończoność.
\end{itemize}

\section{Tworzenie okna}
Okno tworzę przy pomocy biblioteki GTK 3.0. Całe tworzenie okna głównego oraz okna tworzenia planszy jest zaimplementowane w module \texttt{window.c}.
\subsection{Ważniejsze Funkcje}
\begin{itemize}
    \item \texttt{show\_alert(char *alert)} - funkcja do wyświetlania okna dialogowego z komunikatem \texttt{alert}
    \item \texttt{get\_move()} - funkcje do obsługi klikania guzików na planszy
    \item \texttt{create\_board()} - wyświetla okienko wyboru planszy
    \item \texttt{create\_ship()} - próbuje ustawić statek na polach zaczynających się od klikniętego
(ta funkcja jest podłączona do przycisków na planszy ustawiania statków)
    \item \texttt{refresh()} - funkcja do pobierania wiadomości z pliku kolejkowego w odpowiednich odstępach czasowych.
W zależności od otrzymanej wiadomości wywołuje odpowiednie czynności (rozpoczęcie nowej gry, sprawdzenie pola na planszy itp.)
    \item \texttt{chage\_button(GtkWidget *button, Status stat)} - zmienia grafikę na przycisku \texttt{button}
w zależności od statusu \texttt{stat}
    \item \texttt{epilog(char option)} - zakończenie gry w zależności od opcji. (\texttt{'w'} - wyrana, \texttt{'L'} - przegrana,
\texttt{'g'} - przeciwnik poddał się, \texttt{'l'} - poddanie się)
    \item \texttt{update\_board(char option)} - aktualizuje planszę w zależności od opcji (\texttt{'m'} - moją,
\texttt{'o'} - przeciwnika, \texttt{'c'} - ustawiania statków)
    \item \texttt{main()} - tworzy główne okno
\end{itemize}

\subsection{Struktury i zmienne}
\begin{itemize}
    \item \texttt{struct statistics\_struct} - przechowuje wskaźniki na obiekty typu GTK\_LABEL służące
do wyświetlania statystyk
    \item \texttt{struct length\_panel} - przechowuje przyciski zmieniające długość stawianego statku
w oknie tworzenia planszy
    \item \texttt{struct orientation\_panel} - przechowuje przyciski zmieniające orientację stawianego
statku w oknie tworzenia planszy
    \item \texttt{bool my\_round} - \texttt{true} jeśli trwa runda gracza
    \item \texttt{bool game\_run} - \texttt{true} jeśli gra jest rozpoczęta
    \item \texttt{bool end\_of\_game} - \texttt{true} jeśli gra się skończyła
\end{itemize}   

\end{document}
