\documentclass[a4paper]{article}
\usepackage[utf8]{inputenc}
\usepackage[polish]{babel}
\usepackage[T1]{fontenc}
\usepackage{indentfirst}
\usepackage{amsmath, amsfonts}
\usepackage[nofoot,hdivide={2cm,*,2cm},vdivide={2cm,*,2cm}]{geometry}
\frenchspacing
\pagestyle{empty}

\author{Franciszek Zdobylak}
\title{\Huge{\bf{Gra w Statki}}\\
				\small projekt na Wstęp do Programowania w C}

\begin{document}
\maketitle

\section{Opis projektu}

Celem mojego projektu jest zaimplementowanie dwuosobowej gry w statki. Każdy użytkownik będzie mógł wybrać grę którą chce kontynuować lub gracza z którym chce rozpocząć nową grę.
Program będzie umożliwiał grę dwóch graczy korzystających z jednego komputera.

\section{Użyte technologie}
W moim projekcie będę korzystał z języka C wraz z biblioteką GTK+.

\section{Moduły}
Program będzie składał się z modułów, które będą się między sobą komunikowały. Oto cztery główne moduły:
\begin{itemize}
	\item Modułu głównego okna
	\item Modułu obsługi plików (komunikacji)
	\item Modułu obsługi rozgrywki
	\item Modułu ustawienia planszy
\end{itemize}

\subsection{Moduł głównego okna}
Jego zadaniem będzie komunikacja z graczem. Wyświetlanie planszy oraz przekazywanie wyborów gracza do reszty modułów.
Główne funkcje:
\begin{itemize}
	\item Wyświetlenie głównego okna programu
	\item Rozpoczynanie nowej gry z wybranym graczem
	\item Kontynuowanie gry z wybranym graczem (jeśli istnieje taka)
	\item Pobieranie ruchu wybranego przez gracza
	\item Aktualizowanie planszy
\end{itemize}

\subsection{Moduł obsługi rozgrywki}
Jego zadaniem będzie obsługa wszystkich wydarzeń do poprawnego prowadzenia rozgrywki.
Główne funkcje:
\begin{itemize}
	\item Sprawdzanie poprawności ruchu
	\item Wykonanie ruchu
	\item Sprawdzenie czy proponowana pozycja statku jest dozwolona
\end{itemize}

\subsection{Moduł obsługi plików (komunikacji)}
Jego zadaniem jest zarządzaniem zapisanymi rozgrywkami.


{\bf UWAGA!!!} Druga opcją jest przechowywanie planszy tylko u użzytkownika, a wysyłanie tylko danych o ruchach.
Jest to o tyle lepsze, że gracze mają mocno utrudnione ,,podglądanie''.

Główne funkcje:
\begin{itemize}
	\item Tworzenie nowej gry
	\item Otwieranie rozpoczętej gry
	\item Usunięcie rozpoczętej gry
	\item Zapisywanie stanu (co skutkuje ,,wysłaniem'' zmian do drugiego gracza)
\end{itemize}

\subsection{Moduł ustawienia planszy}
Jego zadaniem jest ustawienie statków na planszy podczas tworzenia nowej gry.
Uznałem, że jest to jedno z trudniejszych zagadnień tego projektu, dlatego poświęcam na to osobny moduł.
Główne funkcje:
\begin{itemize}
	\item Wyświetlenie okna ustawienia planszy
	\item Pobieranie wyboru gracza
\end{itemize}

\section{Zapisywanie rozgrywki}
Stan rozgrywki będzie zapisywany w pliku o nazwie \emph{id\_rozgrywki.log}.
Inforamcje zawarte w takim pliku:
\begin{enumerate}
	\item Ruchy poszczególnych graczy wykonane wcześniej (w kolejności od najnowszego do najstarszego)
	\item Opis rozmieszczenia statków na planszach
\end{enumerate}
Kolejność jest ważna, ponieważ dane które program potrzebuje są podane od razu na początku.

\section{Planowane usprawnienia}
Program ma spełniać jedynie podstawowe działania. Niektóre funkcje dostępne w wielu współczesnych programach nie będą dostępne od razu.
Funkcje, które planuję dodać w przyszłości:
\begin{itemize}
	\item Automatyczne odświeżanie 
	\item Automatyczne sprawdzanie dostępnych pól do postawienia statku
	\item Podstawowe statystyki graczy
	\item Zmiany reguł gry (np. liczby statków, rozmiaru planszy, etc.)
\end{itemize}

\end{document}
