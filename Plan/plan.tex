\documentclass[a4paper]{article}
\usepackage[utf8]{inputenc}
\usepackage[polish]{babel}
\usepackage[T1]{fontenc}
\usepackage{indentfirst}
\usepackage{amsmath, amsfonts}
\usepackage[nofoot,hdivide={2cm,*,2cm},vdivide={2cm,*,2cm}]{geometry}
\frenchspacing
\pagestyle{empty}

\author{Franciszek Zdobylak}
\title{\Huge{\bf{Gra w Statki}}\\
				\small projekt na Wstęp do Programowania w C}

\begin{document}
\maketitle

\section{Opis projektu}

Celem mojego projektu jest zaimplementowanie dwuosobowej gry w statki. Każdy użytkownik będzie mógł wybrać grę którą chce kontynuować lub gracza z którym chce rozpocząć nową grę.
Program będzie umożliwiał grę dwóch graczy korzystających z jednego komputera.

\section{Użyte technologie}
W moim projekcie będę korzystał z języka C wraz z biblioteką GTK+.

\section{Moduły}
Program będzie składał się z modułów, które będą się między sobą komunikowały. Oto cztery główne moduły:
\begin{itemize}
	\item Modułu głównego okna
	\item Modułu komunikacji
	\item Modułu obsługi rozgrywki
	\item Modułu ustawienia planszy
\end{itemize}

\subsection{Moduł głównego okna}
Jego zadaniem będzie komunikacja z graczem. Wyświetlanie planszy oraz przekazywanie wyborów gracza do reszty modułów.
Główne funkcje:
\begin{itemize}
	\item Wyświetlenie głównego okna programu
	\item Rozpoczynanie nowej gry z wybranym graczem
	\item Pobieranie ruchu wybranego przez gracza
	\item Aktualizowanie planszy
\end{itemize}

\subsection{Moduł obsługi rozgrywki}
Jego zadaniem będzie obsługa wszystkich wydarzeń do poprawnego prowadzenia rozgrywki.
Główne funkcje:
\begin{itemize}
	\item Sprawdzanie poprawności ruchu
	\item Zapisanie statusu pola (U - gdy jeszcze nie znamy, M1\dots5 - gdy poznamy status)
	\item Sprawdzenie czy proponowana pozycja statku jest dozwolona
	\item Sprawdzanie czy statek został trafiny
\end{itemize}

\subsection{Moduł komunikacji}
Jego zadaniem jest przesyłanie ruchu gracza oraz sprawdzenie statusu poprzedniego strzału.

Główne funkcje:
\begin{itemize}
	\item Zapisywanie strzału ze statusem U (unknown)
	\item Aktualizowanie statusu strzału przeciwnika
	\item Sprawdzanie statusu poprzedniego strzału
\end{itemize}

\subsection{Moduł ustawienia planszy}
Jego zadaniem jest ustawienie statków na planszy podczas tworzenia nowej gry.
Uznałem, że jest to jedno z trudniejszych zagadnień tego projektu, dlatego poświęcam na to osobny moduł.
Główne funkcje:
\begin{itemize}
	\item Wyświetlenie okna ustawienia planszy
	\item Pobieranie wyboru gracza
\end{itemize}

\section{Zapisywanie rozgrywki}
Program każdego z graczy będzie pamiętał rozmieszczenie statkó, oraz planszę z zaznaczonymi strzałami.
Każdy strzał będzie zapisywany do pliku/ plików, a następnie sprawdzany przez program przeciwnika i dopisywany status tego strzału.
\{m - pudło, 1\dots5 - dlugosc statku, u - nie znany (jeszcze)\}.

\section{Planowane usprawnienia}
Program ma spełniać jedynie podstawowe działania. Niektóre funkcje dostępne w wielu współczesnych programach nie będą dostępne od razu.
Funkcje, które planuję dodać w przyszłości:
\begin{itemize}
	\item Automatyczne odświeżanie
	\item Automatyczne sprawdzanie dostępnych pól do postawienia statku
	\item Podstawowe statystyki graczy
	\item Zmiany reguł gry (np. liczby statków, rozmiaru planszy, etc.)
\end{itemize}

\end{document}
