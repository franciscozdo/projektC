\documentclass[a4paper]{article}
\usepackage[utf8]{inputenc}
\usepackage[polish]{babel}
\usepackage[T1]{fontenc}
\usepackage{indentfirst}
\usepackage{amsmath, amsfonts}
\usepackage[nofoot,hdivide={2cm,*,2cm},vdivide={2cm,*,2cm}]{geometry}
\frenchspacing
\pagestyle{empty}

\author{Franciszek Zdobylak}
\title{\Huge{\bf{Gra w Statki}}\\
				\small projekt na Wstęp do Programowania w C}

\begin{document}
\maketitle

\section{Opis projektu}

Celem mojego projektu jest zaimplementowanie dwuosobowej gry w statki. Program będzie uruchamiany z nazwą użytkownika (A lub B) jako parametrem.

\section{Interakcja z użytkownikiem}
Program będzie się komunikował z użytkownikiem przez okienko stworzone dzięki bibliotece GTK. Będzie w nim wyświetlana plansza na której będą zaznaczone statki gracza
(wraz z miejscami trafionymi przez przeciwnika) oraz plansza z miejscami w które trafił gracz przeciwny.

\section{Funkcje programu}
\begin{itemize}
	\item Komunikowanie się z bliźniaczym procesem
	\item Pobieranie ruchu wykonanego przez użytkownika (i sprawdzenie poprawności)
	\item Zaznaczanie i wyświetlanie wykonanych ruchów (gracza i przeciwnika) na planszach
\end{itemize}

\section{Moduły}
Program będzie składał się z modułów. Oto cztery główne moduły:
\begin{itemize}
	\item Modułu głównego okna
	\begin{itemize}
		\item wyświetlanie plansza
		\item pobieranie ruchu
		\item aktualizowanie planszy (np. po strzale przeciwnika)
	\end{itemize}
	\item Modułu obsługi rozgrywki
	\begin{itemize}
		\item sprawdzanie poprawności ruchu
		\item sprawdzanie rozmieszczenia statków
		\item tworzenie i aktualizowanie opisu planszy
	\end{itemize}
	\item Modułu komunikacji
	\begin{itemize}
		\item pobieranie inforamcji z potoków (w odpowiednim formacie)
		\item pisanie informacji do potoków (w odpowiednim formacie)
	\end{itemize}
	\item Modułu ustawienia planszy
	\begin{itemize}
		\item ustawianie planszy na początku rozgrywki
	\end{itemize}
\end{itemize}


\end{document}
